
\documentclass[12pt,leqno]{book}
\usepackage{amsmath,amssymb,amsfonts} % Typical maths resource packages
\usepackage{graphicx}                 % Packages to allow inclusion of graphics
\usepackage{color}                    % For creating coloured text and background
\usepackage{hyperref}                 % For creating hyperlinks in cross references
\usepackage{makeidx}                  % For indexing
\usepackage{listings}                 % For code listing
\usepackage{mathpartir}               % For grammars, rules, etc
\usepackage{bcprules}                 % For other kinds of rules
\usepackage{diagrams}                 % For commutative diagrams

\lstloadlanguages{Scala,Java,Haskell,XML,bash,HTML,SQL}

\parindent 1cm
\parskip 0.2cm
\topmargin 0.2cm
\oddsidemargin 1cm
\evensidemargin 0.5cm
\textwidth 15cm
\textheight 21cm

\newtheorem{theorem}{Theorem}[section]
\newtheorem{proposition}[theorem]{Proposition}
\newtheorem{corollary}[theorem]{Corollary}
\newtheorem{lemma}[theorem]{Lemma}
\newtheorem{remark}[theorem]{Remark}
\newtheorem{definition}[theorem]{Definition}


\def\R{\mathbb{ R}}
\def\S{\mathbb{ S}}
\def\I{\mathbb{ I}}

\def\Scala{\texttt{Scala}}
\def\ScalaCheck{\texttt{ScalaCheck}}
\def\Haskell{\texttt{Haskell}}
\def\XML{\texttt{XML}}

\newcommand{\ldb}{[\![}
\newcommand{\rdb}{]\!]}
\newcommand{\ldrb}{(\!(}
\newcommand{\rdrb}{)\!)}
\newcommand{\lliftb}{\langle\!|}
\newcommand{\rliftb}{|\!\rangle}

\newcommand{\lrclr}{(\!*|}
\newcommand{\rrclr}{|\!*)}

\newcommand{\meaningof}[1]{\ldb #1 \rdb}

\makeindex


\title{Pro Scala: Monadic Design Patterns for the Web}

\author{L.G. Meredith  \\
{\small\em \copyright \  Draft date \today }}

 \date{ }
\begin{document}
\lstset{language=Haskell}
\maketitle
 \addcontentsline{toc}{chapter}{Contents}
\pagenumbering{roman}
\tableofcontents
\listoffigures
\listoftables
\chapter*{Preface}\normalsize
  \addcontentsline{toc}{chapter}{Preface}
\pagestyle{plain}
% The book root file {\tt bookex.tex} gives a basic example of how to
% use \LaTeX \ for preparation of a book. Note that all
% \LaTeX \ commands begin with a
% backslash.

% Each
% Chapter, Appendix and the Index is made as a {\tt *.tex} file and is
% called in by the {\tt include} command---thus {\tt ch1.tex} is
% the name here of the file containing Chapter~1. The inclusion of any
% particular file can be suppressed by prefixing the line by a
% percent sign.


%  Do not put an {\tt end{document}} command at the end of chapter files;
% just one such command is needed at the end of the book.

% Note the tag used to make an index entry. You may need to consult Lamport's
% book~\cite{lamport} for details of the procedure to make the index input
% file; \LaTeX \ will create a pre-index by listing all the tagged
% items in the file {\tt bookex.idx} then you edit this into
% a {\tt theindex} environment, as {\tt index.tex}.

The book you hold in your hands, Dear Reader, is not at all what you expected...



\pagestyle{headings}
\pagenumbering{arabic}

\include{chapters/one/ch}
\include{chapters/two/ch}
\include{chapters/three/ch}
\include{chapters/four/ch}
\include{chapters/five/ch}
\include{chapters/six/ch}
\include{chapters/seven/ch}
\include{chapters/eight/ch}
\include{chapters/nine/ch}
\include{chapters/ten/ch}


\documentclass[12pt,leqno]{book}
\usepackage{amsmath,amssymb,amsfonts} % Typical maths resource packages
\usepackage{graphicx}                 % Packages to allow inclusion of graphics
\usepackage{color}                    % For creating coloured text and background
\usepackage{hyperref}                 % For creating hyperlinks in cross references
\usepackage{makeidx}                  % For indexing
\usepackage{listings}                 % For code listing
\usepackage{mathpartir}               % For grammars, rules, etc
\usepackage{bcprules}                 % For other kinds of rules
\usepackage{diagrams}                 % For commutative diagrams

\lstloadlanguages{Scala,Java,Haskell,XML,bash,HTML,SQL}

\parindent 1cm
\parskip 0.2cm
\topmargin 0.2cm
\oddsidemargin 1cm
\evensidemargin 0.5cm
\textwidth 15cm
\textheight 21cm

\newtheorem{theorem}{Theorem}[section]
\newtheorem{proposition}[theorem]{Proposition}
\newtheorem{corollary}[theorem]{Corollary}
\newtheorem{lemma}[theorem]{Lemma}
\newtheorem{remark}[theorem]{Remark}
\newtheorem{definition}[theorem]{Definition}


\def\R{\mathbb{ R}}
\def\S{\mathbb{ S}}
\def\I{\mathbb{ I}}

\def\Scala{\texttt{Scala}}
\def\ScalaCheck{\texttt{ScalaCheck}}
\def\Haskell{\texttt{Haskell}}
\def\XML{\texttt{XML}}

\newcommand{\ldb}{[\![}
\newcommand{\rdb}{]\!]}
\newcommand{\ldrb}{(\!(}
\newcommand{\rdrb}{)\!)}
\newcommand{\lliftb}{\langle\!|}
\newcommand{\rliftb}{|\!\rangle}

\newcommand{\lrclr}{(\!*|}
\newcommand{\rrclr}{|\!*)}

\newcommand{\meaningof}[1]{\ldb #1 \rdb}

\makeindex


\title{Pro Scala: Monadic Design Patterns for the Web}

\author{L.G. Meredith  \\
{\small\em \copyright \  Draft date \today }}

 \date{ }
\begin{document}
\lstset{language=Haskell}
\maketitle
 \addcontentsline{toc}{chapter}{Contents}
\pagenumbering{roman}
\tableofcontents
\listoffigures
\listoftables
\chapter*{Preface}\normalsize
  \addcontentsline{toc}{chapter}{Preface}
\pagestyle{plain}
% The book root file {\tt bookex.tex} gives a basic example of how to
% use \LaTeX \ for preparation of a book. Note that all
% \LaTeX \ commands begin with a
% backslash.

% Each
% Chapter, Appendix and the Index is made as a {\tt *.tex} file and is
% called in by the {\tt include} command---thus {\tt ch1.tex} is
% the name here of the file containing Chapter~1. The inclusion of any
% particular file can be suppressed by prefixing the line by a
% percent sign.


%  Do not put an {\tt end{document}} command at the end of chapter files;
% just one such command is needed at the end of the book.

% Note the tag used to make an index entry. You may need to consult Lamport's
% book~\cite{lamport} for details of the procedure to make the index input
% file; \LaTeX \ will create a pre-index by listing all the tagged
% items in the file {\tt bookex.idx} then you edit this into
% a {\tt theindex} environment, as {\tt index.tex}.

The book you hold in your hands, Dear Reader, is not at all what you expected...



\pagestyle{headings}
\pagenumbering{arabic}

\include{chapters/one/ch}
\include{chapters/two/ch}
\include{chapters/three/ch}
\include{chapters/four/ch}
\include{chapters/five/ch}
\include{chapters/six/ch}
\include{chapters/seven/ch}
\include{chapters/eight/ch}
\include{chapters/nine/ch}
\include{chapters/ten/ch}


\documentclass[12pt,leqno]{book}
\usepackage{amsmath,amssymb,amsfonts} % Typical maths resource packages
\usepackage{graphicx}                 % Packages to allow inclusion of graphics
\usepackage{color}                    % For creating coloured text and background
\usepackage{hyperref}                 % For creating hyperlinks in cross references
\usepackage{makeidx}                  % For indexing
\usepackage{listings}                 % For code listing
\usepackage{mathpartir}               % For grammars, rules, etc
\usepackage{bcprules}                 % For other kinds of rules
\usepackage{diagrams}                 % For commutative diagrams

\lstloadlanguages{Scala,Java,Haskell,XML,bash,HTML,SQL}

\parindent 1cm
\parskip 0.2cm
\topmargin 0.2cm
\oddsidemargin 1cm
\evensidemargin 0.5cm
\textwidth 15cm
\textheight 21cm

\newtheorem{theorem}{Theorem}[section]
\newtheorem{proposition}[theorem]{Proposition}
\newtheorem{corollary}[theorem]{Corollary}
\newtheorem{lemma}[theorem]{Lemma}
\newtheorem{remark}[theorem]{Remark}
\newtheorem{definition}[theorem]{Definition}


\def\R{\mathbb{ R}}
\def\S{\mathbb{ S}}
\def\I{\mathbb{ I}}

\def\Scala{\texttt{Scala}}
\def\ScalaCheck{\texttt{ScalaCheck}}
\def\Haskell{\texttt{Haskell}}
\def\XML{\texttt{XML}}

\newcommand{\ldb}{[\![}
\newcommand{\rdb}{]\!]}
\newcommand{\ldrb}{(\!(}
\newcommand{\rdrb}{)\!)}
\newcommand{\lliftb}{\langle\!|}
\newcommand{\rliftb}{|\!\rangle}

\newcommand{\lrclr}{(\!*|}
\newcommand{\rrclr}{|\!*)}

\newcommand{\meaningof}[1]{\ldb #1 \rdb}

\makeindex


\title{Pro Scala: Monadic Design Patterns for the Web}

\author{L.G. Meredith  \\
{\small\em \copyright \  Draft date \today }}

 \date{ }
\begin{document}
\lstset{language=Haskell}
\maketitle
 \addcontentsline{toc}{chapter}{Contents}
\pagenumbering{roman}
\tableofcontents
\listoffigures
\listoftables
\chapter*{Preface}\normalsize
  \addcontentsline{toc}{chapter}{Preface}
\pagestyle{plain}
% The book root file {\tt bookex.tex} gives a basic example of how to
% use \LaTeX \ for preparation of a book. Note that all
% \LaTeX \ commands begin with a
% backslash.

% Each
% Chapter, Appendix and the Index is made as a {\tt *.tex} file and is
% called in by the {\tt include} command---thus {\tt ch1.tex} is
% the name here of the file containing Chapter~1. The inclusion of any
% particular file can be suppressed by prefixing the line by a
% percent sign.


%  Do not put an {\tt end{document}} command at the end of chapter files;
% just one such command is needed at the end of the book.

% Note the tag used to make an index entry. You may need to consult Lamport's
% book~\cite{lamport} for details of the procedure to make the index input
% file; \LaTeX \ will create a pre-index by listing all the tagged
% items in the file {\tt bookex.idx} then you edit this into
% a {\tt theindex} environment, as {\tt index.tex}.

The book you hold in your hands, Dear Reader, is not at all what you expected...



\pagestyle{headings}
\pagenumbering{arabic}

\include{chapters/one/ch}
\include{chapters/two/ch}
\include{chapters/three/ch}
\include{chapters/four/ch}
\include{chapters/five/ch}
\include{chapters/six/ch}
\include{chapters/seven/ch}
\include{chapters/eight/ch}
\include{chapters/nine/ch}
\include{chapters/ten/ch}


\documentclass[12pt,leqno]{book}
\usepackage{amsmath,amssymb,amsfonts} % Typical maths resource packages
\usepackage{graphicx}                 % Packages to allow inclusion of graphics
\usepackage{color}                    % For creating coloured text and background
\usepackage{hyperref}                 % For creating hyperlinks in cross references
\usepackage{makeidx}                  % For indexing
\usepackage{listings}                 % For code listing
\usepackage{mathpartir}               % For grammars, rules, etc
\usepackage{bcprules}                 % For other kinds of rules
\usepackage{diagrams}                 % For commutative diagrams

\lstloadlanguages{Scala,Java,Haskell,XML,bash,HTML,SQL}

\parindent 1cm
\parskip 0.2cm
\topmargin 0.2cm
\oddsidemargin 1cm
\evensidemargin 0.5cm
\textwidth 15cm
\textheight 21cm

\include{local/local}

\makeindex


\title{Pro Scala: Monadic Design Patterns for the Web}

\author{L.G. Meredith  \\
{\small\em \copyright \  Draft date \today }}

 \date{ }
\begin{document}
\lstset{language=Haskell}
\maketitle
 \addcontentsline{toc}{chapter}{Contents}
\pagenumbering{roman}
\tableofcontents
\listoffigures
\listoftables
\chapter*{Preface}\normalsize
  \addcontentsline{toc}{chapter}{Preface}
\pagestyle{plain}
% The book root file {\tt bookex.tex} gives a basic example of how to
% use \LaTeX \ for preparation of a book. Note that all
% \LaTeX \ commands begin with a
% backslash.

% Each
% Chapter, Appendix and the Index is made as a {\tt *.tex} file and is
% called in by the {\tt include} command---thus {\tt ch1.tex} is
% the name here of the file containing Chapter~1. The inclusion of any
% particular file can be suppressed by prefixing the line by a
% percent sign.


%  Do not put an {\tt end{document}} command at the end of chapter files;
% just one such command is needed at the end of the book.

% Note the tag used to make an index entry. You may need to consult Lamport's
% book~\cite{lamport} for details of the procedure to make the index input
% file; \LaTeX \ will create a pre-index by listing all the tagged
% items in the file {\tt bookex.idx} then you edit this into
% a {\tt theindex} environment, as {\tt index.tex}.

The book you hold in your hands, Dear Reader, is not at all what you expected...



\pagestyle{headings}
\pagenumbering{arabic}

\include{chapters/one/ch}
\include{chapters/two/ch}
\include{chapters/three/ch}
\include{chapters/four/ch}
\include{chapters/five/ch}
\include{chapters/six/ch}
\include{chapters/seven/ch}
\include{chapters/eight/ch}
\include{chapters/nine/ch}
\include{chapters/ten/ch}

\include{bibliography/monadic}

%\include{index/index}
  \addcontentsline{toc}{chapter}{Index}
\end{document}


%\include{index/index}
  \addcontentsline{toc}{chapter}{Index}
\end{document}


%\include{index/index}
  \addcontentsline{toc}{chapter}{Index}
\end{document}


%\include{index/index}
  \addcontentsline{toc}{chapter}{Index}
\end{document}
